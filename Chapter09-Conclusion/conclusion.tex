
\chapter{Conclusion}
\label{ch:conclusion}

In this dissertation, we explored haptic experience design (\haxd), looking at how to design, build, and evaluate creativity tools supporting \haxd, and looking at the process of haptic experience design.

In this section, we conclude with our findings: A description of the haptic design process, including processes and tasks, strategies, and challenges facing haptic designers, and prescriptive implication for \haxd tools.

\section{Description of the \haxd Process}

\subsection{Processes and tasks}

\subsection{Strategies}

\subsection{Challenges}




\section{Prescriptive Implications for \haxd Tools}

\subsection{Major Requirements and Features}

Real time - focus through to step back (portray as gradient?)

``soft" features - copy and paste, undo/redo, grouping ... mature tool stuff

repetition

Browse/sketch/refine/share


\subsection{Considerations for Implementation}

Web - more people

Track based - flexible, can apply to many different paradigms




\subsection{Evaluating \haxd Tools}



\endinput
