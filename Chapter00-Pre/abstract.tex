%% The following is a directive for TeXShop to indicate the main file
%%!TEX root = diss.tex

\chapter{Abstract}

Haptic technology, which engages the sense of touch, offers promising benefits for a variety of interactions including low-attention displays, emotional connections, and augmented media experiences.
Despite these advantages and an increasing presence of physical devices in commercial and research applications, there is still little support for the design of engaging haptic sensations.
Previous literature has focused on the significant challenges of technological capabilities or physical realism, with limited development on supporting experience design. % or building design tools.

In this dissertation, we seek to answer \textbf{how we can design, build, and evaluate interactive software to support haptic experience design (\haxd).}
We define \haxd and iteratively design three vibrotactile authoring tools, each a case study covering a different user population, vibrotactile device, and design challenge, and use them to observe specific aspects of \haxd with their target users.
%This will result in guidelines for tools that can support vibrotactile design, suggesting designer process and major tool requirements.
We make these in-depth findings more robust in two ways: generalizing results to a breadth of use cases with focused design projects, and grounding them with expert haptic designers through interviews and a workshop.
We organize our findings in two ways: we 1) describe \haxd, including processes, strategies, and challenges, to understand requirements; and 2) develop guidelines on designing, building, and evaluating interactive software that facillitates \haxd.

When characterizing the \haxd process, strategies, and challenges, 
we show that experience design is already practiced with haptic technology, but faces unique considerations compared to other modalities. 
We identify four design activities that must be explicitly supported: browsing, sketching, refining, and sharing.
We find and develop strategies to accommodate the wide variety of haptic devices.
We encapsulate approaches for designing meaning with haptic experiences, and finally, highlight a need for supporting customization and adaptable interfaces.

We then provide directions for designing and implementing, and evaluating \haxd tools.
We discuss the need for improved online deployment and community support.
We present steps to developing research software into mature a \haxd suite of tools.
We comment on important software development points for software support tools, and reflect upon our approaches for evaluating \haxd tools.

By capturing haptic experience design and creating guidelines for supportive tools, we hope to make a first step towards establishing haptic experience design as its own field, akin to graphic and sound design.

%Note: at \ac{UBC}, both the \ac{FoGS} Ph.D. defence programme and the
%Library's online submission system restricts abstracts to 350
%words.

% Embed version information inline -- you should remove this from your
% dissertation
\vfill
\begin{center}
\begin{sf}
\fbox{Revision: r0.7}
\end{sf}
\end{center}
