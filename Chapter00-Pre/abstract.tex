%% The following is a directive for TeXShop to indicate the main file
%%!TEX root = diss.tex

\chapter{Abstract}

Haptic technology, which engages the sense of touch, offers promising benefits for a variety of interactions including low-attention displays, emotional connections, and augmented media experiences.
Despite these advantages and an increasing presence of physical devices in commercial and research applications, there is still little support for the \emph{design} of engaging haptic sensations.
Previous literature has focused on the significant challenges of technological capabilities or physical realism, with limited development on supporting experience design. % or building design tools.

In this dissertation, I ask the following question: \textbf{how can we design, build, and evaluate interactive software to support haptic experience design (HaXD)?}
I have two goals: 1) \emph{describe} HaXD, including processes, strategies, and challenges, to understand requirements; and 2) \emph{prescribe} guidelines on designing, building, and evaluating interactive software that facillitates HaXD.
To accomplish these goals, I will iteratively design three vibrotactile authoring tools, each a case study covering a different user population, vibrotactile device, and design challenge, and use them to observed HaXD with their target users.
%This will result in guidelines for tools that can support vibrotactile design, suggesting designer process and major tool requirements.
I then plan to make these in-depth findings more robust in two ways: generalizing results to a breadth of use cases with side-projects, and grounding them with expert haptic designers through interviews and a workshop.
By capturing haptic experience design and creating guidelines for supportive tools, I hope to make a first step towards establishing haptic experience design as its own field, akin to graphic and sound design.

%Note: at \ac{UBC}, both the \ac{FoGS} Ph.D. defence programme and the
%Library's online submission system restricts abstracts to 350
%words.

% Embed version information inline -- you should remove this from your
% dissertation
\vfill
\begin{center}
\begin{sf}
\fbox{Revision: r0.7}
\end{sf}
\end{center}
