%% The following is a directive for TeXShop to indicate the main file
%%!TEX root = diss.tex

\chapter{Abstract}

Haptic technology, which engages the sense of touch, offers promising benefits for a variety of interactions including low-attention displays, emotional\revRG{ly-aware interfaces}, and augmented media experiences.
Despite %these advantages and 
an increasing presence of physical devices in commercial and research applications, there is still little support for the design of engaging haptic sensations.
Previous literature has focused on the significant challenges of technological capabilities or physical realism rather than on supporting experience design. % or building design tools.

In this dissertation, we study how to design, build, and evaluate interactive software to support haptic experience design (\haxd).
We define \haxd and iteratively design three vibrotactile effect authoring tools, each a case study covering a different user population, vibrotactile device, and design challenge, and use them to observe specific aspects of \haxd with their target users.
%This will result in guidelines for tools that can support vibrotactile design, suggesting designer process and major tool requirements.
We make these in-depth findings more robust in two ways: generalizing results to a breadth of use cases with focused design projects, and grounding them with expert haptic designers through interviews and a workshop.
%We organize our findings in two ways: we
Our findings 1) describe \haxd, including processes, strategies, and challenges; and 2) present guidelines on designing, building, and evaluating interactive software that facillitates \haxd.

When characterizing \haxd processes, strategies, and challenges, 
we show that experience design is already practiced with haptic technology, but faces unique considerations compared to other modalities. 
We identify four design activities that must be explicitly supported: \emph{sketching}, \emph{refining}, \emph{browsing}, and \emph{sharing}.
We find and develop strategies to accommodate the wide variety of haptic devices.
We \revRG{articulate} %encapsulate
approaches for designing meaning with haptic experiences, and finally, highlight a need for supporting adaptable interfaces.

When informing the design, implementation, and evaluation of \haxd tools,
we discover critical features, including a need for improved online deployment and community support.
We present steps to develo\revRG{p} %ing 
\revRG{both existing and future} research software into a mature \revRG{suite of \haxd} %suite of
tools, and reflect upon evaluation methods.
%
By characterizing \haxd and informing supportive tools, we make a first step towards establishing \haxd as its own field, akin to graphic and sound design.

%Note: at \ac{UBC}, both the \ac{FoGS} Ph.D. defence programme and the
%Library's online submission system restricts abstracts to 350
%words.

%% Embed version information inline -- you should remove this from your
%% dissertation
%\vfill
%\begin{center}
%\begin{sf}
%\fbox{Revision: \input{version}}
%\end{sf}
%\end{center}
