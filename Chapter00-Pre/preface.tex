%% The following is a directive for TeXShop to indicate the main file
%%!TEX root = diss.tex

\chapter{Preface}
No creative work occurs by a lone individual; this dissertation is no exception.
All of the projects described in this work are collaborative efforts in at least some capacity.
Even where the author contributed all work, there was often informal feedback from friends, family, and colleagues.
As such, this dissertation will use the first-person plural, ``we", throughout.
In this preface, we clarify the author's contribution to the work, much of which has been published.
%Co-authors names are listed in order of publication contribution where applicable, and alphabetical order otherwise.

In Chapters \ref{ch:introduction}, \ref{ch:rw}, and \ref{ch:conclusion}, Oliver contributed writing and framing, with feedback provided by the supervisor (Dr. Karon MacLean) and supervisory committee (Drs. Ronald Garcia and Michiel van de Panne) throughout his PhD program. 
Some of this thinking \osC{where?} is combined with a handbook chapter currently under review, written with Dr. MacLean as lead other, with Oliver and PhD candidate Hasti Seifi as co-authors.
This chapter is aimed as an advanced (\ie, graduate or senior undergraduate) educational resource incorporating Oliver and Hasti's research.

In \autoref{ch:hapticinstrument}, Oliver contributed all work and ideas, with feedback and guidance from supervisor Dr. Karon MacLean.
The software has been released as an open-source project at \url{https://github.com/ubcspin/mHIVE}.
This work has been published as full conference paper with an associated demo at HAPTICS'14, and at a workshop at CHI'14:

\inlineCitation{Schneider2014}
\inlineCitation{Schneider2014mHIVEdemo}
\inlineCitation{Schneider2014b}

\noindent
In \autoref{ch:tactileanimation}, Oliver contributed most work and ideas, with initial interviews with designers and haptic experts conducted by Disney Research.
This work was conducted while on internship at Disney Research Pittsburgh, with some supplementary work done at UBC.
Dr. Ali Israr supervised Oliver's internship; Oliver led writing with feedback and guidance from Drs. Israr and MacLean.
This work was presented by Oliver at UIST'15 with an associated demo:

\inlineCitation{Schneider2015}
\inlineCitation{Schneider-demo-ta2015}

\noindent
In \autoref{ch:macaron}, Oliver contributed all work and ideas, with feedback and guidance from Dr. MacLean.
Macaron has been released as an open-source project at \url{https://github.com/ubcspin/Macaron} and is available online at \url{http://hapticdesign.github.io/macaron}.
Subsequent development of the core Macaron tool and extension MacaronMix includes work by Matthew Chun, Benson Li, Ben Clark, and Paul Bucci.
The study reported in \autoref{ch:macaron} was presented by Oliver at HAPTICS'16 with an associated demo: 

\inlineCitation{Schneider2016macaron}
\inlineCitation{Schneider-demo-macaronHS2016}


\noindent
In \autoref{ch:hapturk}, Oliver was part of a collaborative team together with PhD candidate Hasti Seifi, undergraduate summer student Matthew Chun, and master's student Salma Kashani, all supervised by Dr. MacLean.
Oliver and Hasti planned and managed the project, with Matthew and Salma doing proxy design, study design, and data collection for low-fidelity proxies and visual proxies respectively.
Oliver lead paper writing and quantitative analysis, working closely with the other authors, and presented the work at CHI'16:

\inlineCitation{Schneider2016hapturk}

\noindent
\autoref{ch:applications} describes several focused projects to give this dissertation improved breadth.
Oliver played different roles depending on the project.
\begin{description}

\item[\autoref{sec:applications:feelcraft}, FeelCraft] Oliver worked closely with Siyan Zhao, supervised by Dr. Israr at Disney Research Pittsburgh.
Oliver implemented the rendering system (co-developed with the engine described in \autoref{ch:tactileanimation}), developed the MineCraft plugin and connection architecture, and wrote the AsiaHaptics paper (archived in LNEE 277) with feedback from Ali Israr.
Artistic contributions to the video were made by Kyna McIntosh and Madeleine Varner.
Oliver and Siyan together designed the implemented feel effects (Oliver lead implementation), planned, shot, and edited the video submissions (Siyan lead editing); each presented the demo once (Oliver at AsiaHaptics 2014, Siyan at UIST 2014):

\inlineCitation{SchneiderAsiaHaptics2014}
\inlineCitation{Schneider-demo-feelcraftUIST2014}


\item[\autoref{sec:applications:feelmessenger}, Feel Messenger] Oliver worked closely with Siyan Zhao and Dr. Israr.
All three developed the concept.
Siyan lead poster design and assisted with figures.
Dr. Israr lead writing assisted by Oliver and Siyan, and presented this work at CHI'15;
Oliver designed and implemented the Feel Messenger application, conducted part of the preliminary study, and lead the demo submission and presentation at World Haptics 2015:

\inlineCitation{Israr2015}
\inlineCitation{Schneider-demo-feelmessenger2015}

\item[\autoref{sec:applications:roughsketch}, RoughSketch] Oliver was the senior graduate student on a four-person student team including Paul Bucci, Gordon Minaker, and Brenna Li.
All four contributed ideas and haptic designs and iteratively developed the final submission.
Paul lead graphic design efforts; Gordon and Brenna presented the work at World Haptics 2015.

\item[\autoref{sec:applications:handson}, HandsOn] Oliver helped supervise Gordon Minaker during a summer NSERC placement and directed studies, with Dr. MacLean supervising and PhD student Richard Davis collaborating.
This work was part of a larger collaborative effort including Melisa Orta Martinez, Dr. Allison Okamura, and Dr. Paulo Blikstein from Stanford University.
Gordon lead the system design and implementation, study design, facilitation, and analysis, and paper writing and submission.
Oliver helped supervise Gordon throughout this process, assisted and supervised by Dr. MacLean.
Richard helped plan the study, write the paper, and provide insights for study implementation.
All three assisted with poster design; Dr. MacLean presented the work at EuroHaptics 2016; the system was also included in a demo presented by Melisa at HAPTICS'16:

\inlineCitation{Minaker2016}
\inlineCitation{Martinez-demo-hapkit2016}


\item[\autoref{sec:applications:cuddlebit}, CuddleBit Design Tools] Oliver collaborated closely with undergraduate David Marino and master's student Paul Bucci, supervised by Dr. MacLean and with support from Hasti Seifi.
Oliver supervised David through his directed studies project, and helped worked with Paul and David in developing and designing the Voodle system.
Oliver worked with Paul Bucci to extend Macaron into MacaronBit and contributed writing to a demo presented at EuroHaptics 2016 by Dr. MacLean:

\inlineCitation{Bucci2016}

\end{description}

\noindent
In \autoref{ch:hapticianinterviews}, UBC alumnus Dr. Colin Swindells conducted interviews and developed interview notes and initial analysis ideas in 2012, supervised by Dr. MacLean and Dr. Kellogg Booth.
In 2015-2016, Oliver transcribed and analyzed the collected interviews, organized and analyzed the \haxd '15 workshop (\url{oliverschneider.ca/HaXD/}) with guidance from Dr. MacLean, and lead writing of a manuscript.
Drs. MacLean and Booth contributed to writing; Dr. Swindells provided feedback.
%As of this writing, this manuscript has been prepared for peer review.





%\subsection{Breadth: Focused Haptic Design Projects}
%Each case study provides concrete knowledge for building a vibrotactile authoring tool, and some insight into the vibrotactile design process.
%However, haptic technology consists of many devices and experiences beyond vibrotactile.
%
%\begin{description}
%	\item[FeelCraft and Feel Messenger] are collaborations with Disney Research members Ali Israr and Siyan Zhao, looking at distributing and customizing haptic effects in a consumer setting with low-fidelity rumble motor devices.
%	I take a haptic designer role to gain a personal understanding of the process, and a software engineer role to understand relevant architectures. % for distributing haptic media.
%
%	\item[CyberHap] is a collaboration between UBC and Stanford looking at force-feedback devices in education; a large team is involved with undergraduate Gordon Minaker leading development of a teaching interface since February 2015, co-supervised by PI Dr. Karon MacLean and me.
%%	I co-supervise Gordon with PI Dr. Karon MacLean in this project looking at force-feedback devices in an education setting.
%	
%	\item[CuddleBit] is a project inspired by the Haptic Creature and CuddleBot project. A small, breathing and vibrating robot will be designed along with a behaviour prototyping tool in summer 2015.
%	I supervise undergraduate Paul Bucci in this project exploring multiple modalities and potential for receiving input through a sensor.
%
%	\item[HapTurk] is a collaboration with PhD candidate Hasti Seifi on different techniques to crowdsource feedback on VT icons. Master's student Salma Kashani and undergraduate Matthew Chun are developing visualizations and low-fidelity VT icons during summer 2015.
%%	This project looks at a formal mechanism for doing large-scale feedback, and tackles the problem of cross-device and cross-modal equivalency (can a sensation be rendered in some fidelity on both low- and high-fidelity devices?).
%
%	\item[RoughSketch] is a painting application for the TPad Phone, a variable-friction mobile device, for the World Haptics 2015 Student Innovation Challenge. Undergraduates Brenna Li, Paul Bucci, and Gordon Minaker are all fellow team members. Variable friction is a significant contrast to VT sensations as it is intrinsically connected to input: no sensation can be felt without active movement by the user.
%\end{description}
