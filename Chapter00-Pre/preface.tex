%% The following is a directive for TeXShop to indicate the main file
%%!TEX root = diss.tex

\chapter{Preface}
No creative work occurs by a lone individual; this dissertation is no exception.
All of the projects described in this work are collaborative efforts in at least some capacity.
Even where the author contributed all work, there was often informal feedback from friends, family, and lab mates.
As such, this dissertation will use the first-person plural, ``we", throughout.
In this preface, we clarify the author's contribution to the work, much of which has been published.
%Co-authors names are listed in order of publication contribution where applicable, and alphabetical order otherwise.

In \autoref{ch:introduction}, \autoref{ch:rw}, and \autoref{ch:conclusion}, Oliver contributed writing and framing, with feedback provided by the supervisor (Dr. Karon MacLean) and supervisory committee (Drs. Ronald Garcia and Michiel van de Panne) throughout his PhD program. 
Some of this thinking is combined with a handbook chapter currently under review, developed with Dr. Karon MacLean and Hasti Seifi: TODO.

In \autoref{ch:hapticinstrument}, Oliver contributed all work and ideas, with feedback and influence from supervisor Dr. Karon MacLean.
This work has been published as \bibentry{Schneider2014}.

in \autoref{ch:tactileanimation}, Oliver contributed all work and ideas, other than initial interviews with designers and haptic experts. 
This work was conducted while on internship at Disney Research Pittsburgh, with some supplementary work done at UBC, with feedback and influence from internship supervisor Dr. Ali Israr and PhD supervisor Dr. Karon MacLean, and has been published as \bibentry{Schneider2015}.

In \autoref{ch:macaron}, Oliver contributed all work and ideas, with feedback and influence from supervisor Dr. Karon MacLean, except for subsection TODO, where additional development work was done by TODO, TODO, and TODO.
The majority of this work has been published as \bibentry{Schneider2016macaron}; Section TODO is currently in development for a peer-reviewed submission.

In \autoref{ch:hapturk}, Oliver was part of a collaborative team together with PhD student Hasti Seifi, undergraduate summer student Matthew Chun, and master's student Salma Kashani, all supervised by Dr. Karon MacLean.
Oliver and Hasti planned and managed the project, with Matthew and Salma doing proxy design, study design, and data collection for low-fidelity proxies and visual proxies respectively.
Oliver lead paper writing and quantitative analysis, working closely with the other authors, and presented the work published as \bibentry{Schneider2016hapturk}.

In \autoref{ch:applications}, Oliver played different roles depending on the focused design project.
\begin{description}
\item[Section TODO, FeelCraft] Oliver worked closely with Siyan Zhao, supervised by Ali Israr.
Oliver implemented the rendering system (which was co-developed with the engine described in \autoref{ch:tactileanimation}), developed the MineCraft plugin and connection architecture, and wrote the AsiaHaptics paper with feedback from Ali Israr.
Oliver and Siyan together designed the implemented feel effects (Oliver lead implementation), planned, shot, and edited the video submissions (Siyan lead editing); each presented the demo once (Oliver at AsiaHaptics 2014, Siyan at UIST 2014).
Artistic contributions to the video were made by Kyna McIntosh and Madeleine Varner.
This work has been published as \bibentry{SchneiderAsiaHaptics2014}.

\end{description}







%\subsection{Breadth: Focused Haptic Design Projects}
%Each case study provides concrete knowledge for building a vibrotactile authoring tool, and some insight into the vibrotactile design process.
%However, haptic technology consists of many devices and experiences beyond vibrotactile.
%
%\begin{description}
%	\item[FeelCraft and Feel Messenger] are collaborations with Disney Research members Ali Israr and Siyan Zhao, looking at distributing and customizing haptic effects in a consumer setting with low-fidelity rumble motor devices.
%	I take a haptic designer role to gain a personal understanding of the process, and a software engineer role to understand relevant architectures. % for distributing haptic media.
%
%	\item[CyberHap] is a collaboration between UBC and Stanford looking at force-feedback devices in education; a large team is involved with undergraduate Gordon Minaker leading development of a teaching interface since February 2015, co-supervised by PI Dr. Karon MacLean and me.
%%	I co-supervise Gordon with PI Dr. Karon MacLean in this project looking at force-feedback devices in an education setting.
%	
%	\item[CuddleBit] is a project inspired by the Haptic Creature and CuddleBot project. A small, breathing and vibrating robot will be designed along with a behaviour prototyping tool in summer 2015.
%	I supervise undergraduate Paul Bucci in this project exploring multiple modalities and potential for receiving input through a sensor.
%
%	\item[HapTurk] is a collaboration with PhD candidate Hasti Seifi on different techniques to crowdsource feedback on VT icons. Master's student Salma Kashani and undergraduate Matthew Chun are developing visualizations and low-fidelity VT icons during summer 2015.
%%	This project looks at a formal mechanism for doing large-scale feedback, and tackles the problem of cross-device and cross-modal equivalency (can a sensation be rendered in some fidelity on both low- and high-fidelity devices?).
%
%	\item[RoughSketch] is a painting application for the TPad Phone, a variable-friction mobile device, for the World Haptics 2015 Student Innovation Challenge. Undergraduates Brenna Li, Paul Bucci, and Gordon Minaker are all fellow team members. Variable friction is a significant contrast to VT sensations as it is intrinsically connected to input: no sensation can be felt without active movement by the user.
%\end{description}
